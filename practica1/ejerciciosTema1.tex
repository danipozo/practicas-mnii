\documentclass[11pt]{article}

 \usepackage[spanish]{babel}
 \usepackage[utf8]{inputenc}
 \usepackage{amssymb}
 \usepackage{amsmath}
 \usepackage{verbatim}
 \usepackage{systeme}
 \usepackage{array}

\title{\textbf{Ejercicios Tema 1}}
\author{Laura Gómez, Javier Sáez, Daniel Pozo, Luis Ortega}

\begin{document}

\maketitle
\section{Ejercicio 7}
\textbf{Enunciado}: Se considera la ecuación:
\[
x + \log x = 0
\]
\begin{itemize}
	\item Prueba que dicha ecuación posee una única solución
	\item Sea $a \in (0, 1/2)$. Prueba que si $x_0 \in [a,1]$, el método de Newton-Raphson es convergente.
\end{itemize}
\textbf{Resolución}:\\
\begin{itemize}
	\item Si $f(x) = x + \log x$, entonces $f'(x) = 1 + \frac{1}{x}$. Veremos cuándo esta derivada es igual a 0.
	\[
	1+\frac{1}{x} = 0 \iff x = -1.
	\]
	Ahora, la función $f(x)$ está definida únicamente en el intervalo $(0,+\infty)$, y es una función creciente en todo su dominio, luego en caso de existir la solución, a partir de ahora conocida como $b$, será única. Además, sabemos que existe y que $b\in(1/2,1)$ pues, al aplicar Bolzano, $f(1/2) < 0$ y $f(1) > 0$.
	
	\item Sabemos que $f'(x)>0 \ \ \forall x \in \mathbb{R}^+$. Además, $f\in \mathcal{C}^2(\mathbb{R}^+)$, luego $f\in \mathcal{C}^2([a,1]) \ \ \forall a \in (0,1/2)$. Nótese que $f(a) < 0 \ \ \forall a \in (0,1/2)$ y que $f(1) > 0$. Tenemos así lo siguiente:
	\begin{enumerate}
	\item $f \in \mathcal C ^2 (\mathbb R ^+)$
	\item $f(a)f(1) < 0 \ \  \forall a \in (0,1/2)$
	\item $\forall x \in \mathbb R ^+ \ \ f'(x) \ne 0$
	\item $f''(x)= -\frac{1}{x^2}$ es siempre negativa en $\mathbb R ^2$ y por tanto en $[a,1]$.
\end{enumerate}
Luego tenemos las condiciones para aplicar la proposición de la convergencia global del método de Newton. Por ello $\forall a \in (0,\frac{1}{2}), \forall x_0 \in [a,1]$ tal que $h(x_0)=f(x_0)f''(x_0)\geq0$ tenemos que la sucesión converge.

A continuación veremos para qué valores $y$ se cumple lo enunciado.

\[
f(y)f''(y)=(y+\log y)\frac{-1}{y^2}\geq 0\iff y+\log y \leq 0 \iff f(y)\leq 0
\]

Por todo lo que conocemos hasta ahora sobre $f$, como que tiene una única solución llamada $b$ y que es una función creciente, podemos afirmar entonces que si $x_0\in[a,b]$ entonces el método de Newton converge a la raiz $b$. Nuestra pregunta a continuación es: si $x_0\in[b,1]$, ¿sigue convergiendo la ecuación?

Para ello, calcularemos el valor $x_1$ a partir de un $x_0\in[b,1]$ cualquiera pero fijo y veremos qué sucede con él.

\[
x_1=x_0-\frac{f(x_0)}{f'(x_0)}=x_0-\frac{x_0-\log x_0}{1+\frac{1}{x_0}}=\frac{1-\log x_0}{1+\frac{1}{x_0}}=h(x_0)
\]

¿$x_1\in[0,b]$? O lo que es lo mismo, ¿$h:[b,1]\rightarrow A \subset[a,b]$? Veremos cómo es el crecimiento de $h$

\[
h'(x_0)=\frac{-x_0-\log x_0}{x_0+1+2x_0^2}=0 \iff x_0=b\]\[
h'(1)<0
\]

Por ello nuestra $h$ es decreciente, su mínimo se alcanza en 1 y su máximo se alcanza en $b$. Comprobamos que $\forall a\in(0,\frac{1}{2})$ $ x_1 \ge a$:
\[
x_1=h(x_0)=\frac{1-\log x_0}{1+\frac{1}{x_0}}\ge a\iff \frac{1-\log 1}{1+\frac{1}{1}} \geq 1 \iff \frac{1}{2} \ge a 
\]

Comprobamos que $x_1\leq b$:

\[\arraycolsep=2pt\def\arraystretch{2.2}
\begin{array}{>{\displaystyle}ll}
  x_1=h(x_0)=\frac{1-\log x_0}{1+\frac{1}{x_0}}\leq b & \iff \\
  \frac{1-\log b}{1+\frac{1}{b}} \leq b & \iff \\
  \frac{b-b\log b}{b+1} \leq b & \iff \\
  \frac{1-\log b}{b+1} \leq 1
\end{array}
\]

Como no conocemos el valor $b$, veremos cuál el máximo valor de

\[g(b)=\frac{1-\log b}{b+1} \quad \forall b \in \left(\frac{1}{2},1\right)\]

que es el intervalo donde nos hemos asegurado, al inicio del ejercicio, que existe la solución de nuestra ecuación.

Terminando, como $g'(b)<0$ $\forall b \in \left(\frac{1}{2},1\right)$, tenemos que la función es estrictamente decreciente en todo el intervalo y por ello su máximo se alcanza en $\frac{1}{2}$ :
\[\frac{1-\log b}{b+1}\leq \frac{1-\log (\frac{1}{2})}{\frac{1}{2}+1}=0.795431... < 1\]

Como queríamos demostar. Por tanto, $\forall a\in(0,\frac{1}{2})$si $x_0\in[a,1]$ el método de Newton-Raphson $b$ converge siendo por aplicación directa del teorema de convergencia global de Newton si $x_0 \in [a,b]$ y porque $\forall x_0 \in [b,1], x_1\in [a,b]$ y nuevamente se aplica el teorema de la convergencia global de Newton.
\end{itemize}

\section{Ejercicio 10}
\textbf{Enunciado}: Se considera la función: $g(x) = \lambda x(1-x)$, con $\lambda \in [0,4]$.
\begin{itemize}
	\item Demuestra que $g([0,1]) \subset [0,1]$.
	\item Calcula los puntos fijos de la función en $[0,1]$ en función de $\lambda$.
	\item Considera la suceción de iteraciones $x_{n+1} = g(x_n)$, $n=0,1,...$ y analiza la convergencia de dicha sucesión a los puntos fijos de $g$, en función de $\lambda$.
\end{itemize}

\textbf{Solución}:\\
\begin{itemize}
	\item Sabemos que $g$ es una funcion continua, y que $[0,1]$ es un compacto, por tanto $g([0,1])$ tendrá un máximo y un mínimo. Vemos que:
	\[
	g'(x) = \lambda(1-x) - \lambda x  = \lambda(1-2x)
	\]
	Ahora, $g'(x) = 0 \iff x = \frac{1}{2}$, y $g''(x) = -2 \lambda < 0$, luego tiene un máximo en $x=1/2$. $g(1/2) = \frac{\lambda}{4}$. Este máximo es menor o igual que $1$, para todo $\lambda$ que esté entre 0 y 4.\\
	Además, $g(0) = g(1) = 0$, por tanto concluimos que $g([0,1]) \subset [0,1]$
	
	\item Vamos a buscar los puntos fijos. Para ello, se tiene que dar:
	\[
	g(x) = x \implies x = \lambda x(1-x)
	\]
	De esta expresión, obtenemos fácilmente que $x=0$ es un punto fijo. Supongamos $x\ne0$. Entonces,
	\[
	1 = \lambda(1-x) \implies x = 1- \dfrac{1}{\lambda}
	\]
	
	Tenemos que tener entonces que $\lambda \ne 0$. Ahora, tenemos que ver cuándo estos puntos fijos están dentro del intervalo $[0,1]$. Vemos que si $\lambda \in (0,1)$, entonces $\dfrac{1}{\lambda}>1$, por lo que el punto fijo sería $x_\lambda < 0$, y no estaría en nuestro intervalo. Por tanto, para que nuestros puntos fijos se queden en $[0,1]$, necesitamos que $\lambda \in [1,4]$
	
	\item Sea $x_{n+1} = g_\lambda(x) = \lambda x(1-x) = \lambda x - \lambda x^2$ con $\lambda \in [1,4]$ y $x \in [0,1]$. Como sabemos que $g([0,1]) \subset [0,1]$, y sabiendo que los puntos fijos son de la forma $x_\lambda = 1 - \dfrac{1}{\lambda}$, analizaremos la convergencia de dicha sucesión a los puntos fijos $x=x_\lambda$ y $x=0$.

\begin{comment}	
	Vamos a usar el siguiente resultado:

	Sea $[a,b] \subseteq \mathbb R$ un intervalo cerrado y $g:[a,b] \to \mathbb R$ tal que:
	\begin{enumerate}
	\item $g_\lambda(x) \in [a,b] \ \ \ \forall x \in [a,b]$
	\item $g_\lambda$ es lipschitziana con constante de Lipschitz $L<1$, entonces $\exists ! x_\lambda \in [a,b]$ tal que $g(x_\lambda) = x_\lambda$ y $x_\lambda ^* = \lim_{n \to \infty} \{x_{n+1}\} \ \ \forall x_0 \in [a,b]$
\end{enumerate}

Sabemos que la primera condición se verifica $\forall \lambda \in [0,4]$ por el primer apartado. La segunda condición se verifica si $|g_\lambda'(x)| \leq L < 1 \ \ \ \forall x \in [a,b]$.
\end{comment}
Sabiendo que si $|g'_\lambda(x)|<1$ en $x$ punto fijo, entonces este es atractivo. De la misma forma si $|g'_\lambda(x)|>1$ entonces es repulsivo. Analicemos primero el caso del punto fijo $x=0$ con $g_\lambda'(x)= \lambda - 2\lambda x$, luego:

\[g_\lambda'(0) = \lambda - 2\lambda 0=\lambda \implies \forall \lambda \in [1,4], g'_\lambda(0)>0\] luego $x=0$ es un punto fijo repulsivo siempre.

Veamos ahora qué sucede en $x=x_\lambda$ punto fijo. Miremos cuándo $g_\lambda'(x)= \lambda - 2\lambda x$ es, en valor absoluto, es menor o igual que 1 para averiguar su comportamiento en función de $\lambda$:
\[
g_\lambda ' = 1 \implies \lambda - 2\lambda x = 1 \implies x = \frac{1}{2} - \frac{1}{2\lambda}
\]
De forma análoga, $g_\lambda' = -1 $ se da cuando $x = \frac{1}{2} + \frac{1}{2\lambda}$.
	En estos dos puntos, tenemos que el valor absoluto de la derivada es 1. Ahora, esutudiemos el crecimiento y decrecimiento de esta derivada. Para ello, vemos que $g_\lambda''(x) = -2 \lambda$, luego $g_\lambda'$ es decreciente para todo $\lambda \in [1,4]$. Por tanto, $\forall x \in(\frac{1}{2}-\frac{1}{2\lambda},\frac{1}{2}+\frac{1}{2\lambda})$, se tiene que $|g'(x)| < 1$.\\


Comprobamos para qué valores de $\lambda$, nuestro punto fijo $x_\lambda = 1 - \dfrac{1}{\lambda}$ pertenece a dicho intervalo o vemos cuándo $|g'_\lambda(x_\lambda)| = 1$, tendremos que:
	
	\begin{itemize}
	\item Si $\lambda \in (1,3)$, entonces $|g'_\lambda(x_\lambda)| < 1 \implies x_\lambda$ es un punto fijo atractivo
	\item Si $\lambda \in(3,4] \implies |g'_\lambda(x_\lambda)| > 1 \implies x_\lambda$ es un punto fijo repulsivo
	\item Si $\lambda = 1$ tenemos que $|g'_1(x_1)|=|g'_\lambda(0)|$ que, como ya hemos afirmado antes, es repulsivo.
	\item Si $\lambda = 3 \implies |g'_3(x_3)|=|g'_3(\frac{2}{3})|=1$ En este caso, si analizamos el límite por la izquierda y por la derecha de $x_3=\frac{2}{3}$, viéndose que el límite por la izquierda es mayor estricto que 1 y menor estricto que 1 si tomamos el límite por la derecha. De esta forma vemos que, en este caso, los valores a la izquierda del punto fijo se ven atraídos hacia a él y los valores a la derecha se ven repelidos.
	
	\end{itemize}

\end{itemize}
 
\section{Ejercicio 18}
\textbf{Enunciado}:
Se considera el sistema de ecuaciones:
\[
\begin{cases}
	3x_1 - \cos (x_2x_3) - \dfrac{1}{2} = 0\\
	 x_1^2 -81(x_2+0.1)^2 + \sin x_3 +1.06 = 0\\
	 e^{-x_1x_2}+20x_3+\dfrac{10\pi-3}{3} = 0
\end{cases}
\] 
\begin{itemize}
	\item Escribe el sistema anterior en la forma $x=g(x)$ depejando en la ecuación $i$ la variable $x_i$, $i=1,2,3$.
	\item Demuestra utilizando el resultado del ejercicio anterior que el sistema de ecuaciones tiene una única solución en 
	\[
	D = \{(x_1,x_2,x_3) \in \mathbb R^3 \ : \- 1 \leq x_i \leq 1, \ i = 1,2,3\}.
	\]
	
	\item Calcula una aproximación de la solución con el método de iteración funcional tomando $x^{(0)} = (0.1,0.1,-0.1)$ con una tolerancia fijada de $10^{-5}$, donde la tolerancia viene fijada por la norma infinito de la diferencia de dos aproximaciones sucesivas.

\end{itemize}

\textbf{Solución}:

\begin{itemize}
	\item Despejando en nuestro sistema, obtejemos:
	\[
	\displaystyle\begin{cases}
	x_1 = \dfrac{\cos (x_2x_3) + 1/2}{3}\\
 x_2 = 1/90 (9 - \sqrt2 \sqrt{53 + 50 x_1^2 + 50 \sin (x_3)}\ )\\
  x_3 = \dfrac{-e^{-x_1x_2}- \frac{10\pi-3}{3}}{20}
\end{cases}
	\]
	queda así despejado nuestro sistema.
	Ahora, vamos a usar el resultado del ejercicio 17 para probar que tiene una única solución.
	El resultado del ejercicio anterior es:\\
	Si $g:D\subset \mathbb R^n \to \mathbb R^n$ de clase 1 e $D$, si existe $L \in (0,1)$ tal que:
	\[
	\left|\dfrac{\partial g_i(x)}{\partial x_j}\right| \leq \dfrac{L}{n}, \quad \forall x \in D
	\]
	entonces $g$ es contractiva.
	
	Ahora, usaremos que $D$ es un dominio y que $g(x) = (x_1,x_2,x_3)$ es contractiva (lo probaremos ahora) para ver que tiene un único punto $x^*$ atal que $g(x^*) = x^*$.
	
	Vamos por ello a realizar la jacobiana de $g$.
        
        \[
        \setlength{\delimitershortfall}{0pt}
        \begin{pmatrix}
    0 & \dfrac{-x3 \sin (x_2x_3)}{3} & \dfrac{-x2 \sin (x_2x_3)}{3}\\[2ex]
    
  \dfrac{-5 \sqrt2 x}{9 \sqrt{53 + 50 x^2 + 50 \sin (x_3)}}& 0 &  -\dfrac{5 \cos (x)}{9 \sqrt2 \sqrt{53 + 50 x_1^2 + 50 \sin(x)}} \\[2ex]
    
    \dfrac{-x_2e^{-x_1x_2}}{20} & 
    \dfrac{-x_1e^{-x_1x_2}}{20} & 0
  \end{pmatrix}
        \]
        

 Podemos ver que, como $-1 \leq x_i \leq 1$, entonces cada una de estas derivadas es menor que $\dfrac{1}{3}$. Por tanto, como $n=3$, tomando $L$ cercano a 1, tenemos la condición que queríamos. Podemos afirmar así que $g$ es contractiva. Además, $D$ es claramente un dominio, por lo que concluimos que $\exists ! x^* : g(x^*) = x^*$.
\end{itemize}

\section{Ejercicio 21}
\textbf{Enunciado}: Obtén aproximaciones de la solución de los sistemas de los ejercicios 18 y 20 mediante el método de Newton. Compara la convergencia de los resultados obtenidos con los diferentes métodos.
\end{document}
