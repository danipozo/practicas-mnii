\documentclass[11pt]{article}

 \usepackage[spanish]{babel}
 \usepackage[utf8]{inputenc}
 \usepackage{amssymb}
 \usepackage{amsmath}

\usepackage{systeme}

\title{\textbf{Ejercicios Tema 1}}
\author{Laura Gómez, Javier Sáez, Daniel Pozo, Luis Ortega}

\begin{document}

\maketitle
\section{Ejercicio 7}
\textbf{Enunciado}: Se considera la ecuación:
\[
x + logx = 0
\]
\begin{itemize}
	\item Prueba que dicha ecuación posee una única solución
	\item Sea $a \in (0, 1/2)$. Prueba que si $x_0 \in [a,1]$, el método de Newton-Raphson es convergente.
\end{itemize}
\textbf{Resolución}:\\
\begin{itemize}
	\item Si $f(x) = x + logx$, entonces $f'(x) = 1 + \frac{1}{x}$. Veremos cuándo esta derivada es igual a 0.
	\[
	1+\frac{1}{x} = 0 \iff x = -1.
	\]
	Ahora, la función $f(x)$ está definida únicamente en el intervalo $(0,+\infty)$, y es una ufnción creciente en todo su dominio, luego en caso de existir la solución será única. Además, sabemos que existe pues $f(0,1) < 0$ y $f(1) > 0$.
	
	\item Sabemos que $f'(x)>0 \ \ \forall x \in R^+$. Además, $f\in \mathcal{C}^2(\mathbb{R}^+)$, luego $f\in \mathcal{C}^2([a,1]) \ \ \forall a \in (0,1/2)$. Nótese que $f(a) < 0 \ \ \forall a \in (0,1/2)$ y que $f(1) > 0$. Tenemos así lo siguiente:
	\begin{enumerate}
	\item $f \in \mathcal C ^2 (\mathbb R ^+)$
	\item $f(a)f(b) < 0 \ \  \forall a \in \mathbb R ^+$
	\item $\forall x \in \mathbb R ^+ \ \ f'(x) \ne 0$
	\item $f''$ es siempre positiva en $\mathbb R ^2$
\end{enumerate}
Luego tenemos las condiciones para aplicar la proposición de la convergencia global del método de Newton.
\end{itemize}

\section{Ejercicio 10}
\textbf{Enunciado}: Se considera la función: $g(x) = \lambda x(1-x)$, con $\lambda \in [0,4]$.
\begin{itemize}
	\item Demuestra que $g([0,1]) \subset [0,1]$.
	\item Calcula los puntos fijos de la función en $[0,1]$ en función de $\lambda$.
	\item Considera la suceción de iteraciones $x_{n+1} = g(x_n)$, $n=0,1,...$ y analiza la convergencia de dicha sucesión a los puntos fijos de $g$, en función de $\lambda$.
\end{itemize}

\textbf{Solución}:\\
\begin{itemize}
	\item Sabemos que $g$ es una funcion continua, y que $[0,1]$ es un compacto, por tanto $g([0,1])$ tendrá un máximo y un mínimo. Vemos que:
	\[
	g'(x) = \lambda(1-x) - \lambda x  = \lambda(1-2x)
	\]
	Ahora, $g'(x) = 0 \iff x = \frac{1}{2}$, y $g''(x) = -2 \lambda < 0$, luego tiene un máximo en $x=1/2$. $g(1/2) = \frac{\lambda}{4}$. Este máximo es menor o igual que $1$, para todo $\lambda$ que esté entre 0 y 4.\\
	Además, $g(0) = g(1) = 0$, por tanto concluimos que $g([0,1]) \subset [0,1]$
	
	\item Vamos a buscar los puntos fijos. Para ello, se tiene que dar:
	\[
	g(x) = x \implies x = \lambda x(1-x)
	\]
	De esta expresión, obtenemos fácilmente que $x=0$ es un punto fijo. Supongamos $x\ne0$. Entonces,
	\[
	1 = \lambda(1-x) \implies x = 1- \dfrac{1}{\lambda}
	\]
\end{itemize}

\section{Ejercicio 18}
\textbf{Enunciado}:
Se considera el sistema de ecuaciones:
\[
\systeme*{3x_1 - cos(x_2x_3) - \dfrac{1}{2} = 0, x_1^2 -81(x_2+0.1)^2 + sinx_3 +1.06 = 0, e^{-x_1x_2}+20x_3+\dfrac{10\pi-3}{3} = 0}
\] 
\begin{itemize}
	\item Escribe el sistema anterior en la forma $x=g(x)$ depejando en la ecuación $i$ la variable $x_i$, $i=1,2,3$.
	\item Demuestra utilizando el resultado del ejercicio anterior que el sistema de ecuaciones tiene una única solución en 
	\[
	D = \{(x_1,x_2,x_3) \in \mathbb R^3 \ : \- 1 \leq x_i \leq 1, \ i = 1,2,3\}.
	\]
	
	\item Calcula una aproximación de la solución con el método de iteración funcional tomando $x^{(0)} = (0.1,0.1,-0.1)$ con una tolerancia fijada de $10^{-5}$, donde la tolerancia viene fijada por la norma infinito de la diferencia de dos aproximaciones sucesivas.

\end{itemize}

\textbf{Solución}:

\begin{itemize}
	\item Despejando en nuestro sistema, obtejemos:
	\[
	\systeme*{x_1 = \frac{cos(x_2x_3) + 1/2}{3}, x_2 = \sqrt{\frac{x_1^2 -0.81+1.06-16.2x_2+sinx_3}{81}}, x_3 = \frac{-e^{-x_1x_2}- \frac{10\pi-3}{3}}{20}}
	\]
	queda así despejado nuestro sistema.
	Ahora, vamos a usar el resultado del ejercicio 17 para probar que tiene una única solución.
	El resultado del ejercicio anterior es:\\
	Si $g:D\subset \mathbb R^n \to \mathbb R^n$ de clase 1 e $D$, si existe $L \in (0,1)$ tal que:
	\[
	|\dfrac{\partial g_i(x)}{\partial x_j}| \leq \dfrac{L}{n}, \quad \forall x \in D
	\]
	entonces $g$ es contractiva.
	
	Ahora, usaremos que $D$ es un dominio y que $g(x) = (x_1,x_2,x_3)$ es contractiva (lo probaremos ahora) para ver que tiene un único punto $x^*$ atal que $g(x^*) = x^*$.
	
	Vamos por ello a realizar la jacobiana de $g$.
\[	\begin{pmatrix}
    0 & \frac{-x3 sin(x_2x_3)}{3} & \frac{-x2 sin(x_2x_3)}{3}\\
    \frac{x_1}{3\sqrt{x_1^2 + sin(x_3)+0.25-16.2x_2}}& \frac{-8.1}{3\sqrt{x_1^2+ sin(x_3)-16.2x_2+0.25}} & \frac{cos(x_3)}{6\sqrt{x_1^2+16.2x_2+sin(x_3)+0.25}} \\
    \frac{-x_2e^{-x_1x_2}}{20} & 
    \frac{-x_1e^{-x_1x_2}}{20} & 0
  \end{pmatrix}
 \]
 que podemos observar que , como $x_1,x_2,x_3$ está siempre entre 0 y 1, entonces se cumple la condición que necesitamos para aplicar la proposición anunciada con anterioridad y afirmar así que $g$ es contractiva. Además, $D$ es claramente un dominio, por lo que concluimos que $\exists ! x^* : g(x^*) = x^*$.
\end{itemize}

\section{Ejercicio 21}
\textbf{Enunciado}: Obtén aproximaciones de la solución de los sistemas de los ejercicios 18 y 20 mediante el método de Newton. Compara la convergencia de los resultados obtenidos con los diferentes métodos.
\end{document}
