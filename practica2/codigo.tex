\documentclass[11pt]{article}
%Gummi|065|=)
\usepackage[utf8]{inputenc}
\usepackage[spanish]{babel}
\title{\textbf{Código en python de los algoritmos}}
\author{Javier Sáez, Laura Gómez, Daniel Pozo, Luis Ortega}
\date{}


\usepackage{listings}
\usepackage{color}

\definecolor{dkgreen}{rgb}{0,0.6,0}
\definecolor{gray}{rgb}{0.5,0.5,0.5}
\definecolor{mauve}{rgb}{0.58,0,0.82}

\lstset{frame=tb,
  language=Java,
  aboveskip=3mm,
  belowskip=3mm,
  showstringspaces=false,
  columns=flexible,
  basicstyle={\small\ttfamily},
  numbers=none,
  numberstyle=\tiny\color{gray},
  keywordstyle=\color{blue},
  commentstyle=\color{dkgreen},
  stringstyle=\color{mauve},
  breaklines=true,
  breakatwhitespace=true,
  tabsize=3
}

\begin{document}

\maketitle

\section{Programa 1- Newton Cotes}

\begin{lstlisting}
from sympy import Symbol
from sympy import integrate
from decimal import *
"""
    Estos seran los datos de entrada de nuestro programa.
        f(x) representa la funcion que queremos integrar
        [a,b] es el intervalo donde queremos obtener el valor de la integral
        n+1 son los nodos que utilizaremos.
"""
x=Symbol('x')
f = 1/(1+x**2)

a=-4
b=4
n=10

"""
    A partir de aqui, comienza el codigo de nuestro programa.
    Primero de todo, calcularemos el valor de h, los nodos que utilizaremos,
    asi como la base de Lagrange o pesos que utilizaremos para nuestro calculo
    de la integral.
"""

h=Decimal((b-a))/Decimal(n)

nodos=[]
for i in range(n+1):
    nodos.append(a+i*h)

pesos=[]
for i in range(n+1):
    aux=1
    for j in range(n+1):
        if(j != i):
            aux=aux*(x-nodos[j])/(nodos[i]-nodos[j])
            """print(aux)"""
    pesos.append(aux)

"""print(pesos[0])
print(integrate(pesos[0],(x,a,b)))"""

"""
    Una vez que tenemos calculados los pesos, en funcion de x, calcularemos el
    valor de una aproximacion de la integral.
"""
integral=0
for i in range(n+1):
    integral=integral+integrate(pesos[i],(x,a,b))*f.subs(x,nodos[i])

print(integral)
\end{lstlisting}

\section{Programa 2- Newton compuesto}

\begin{lstlisting}
from sympy import Symbol
from sympy import integrate
from decimal import *
"""
    Estos seran los datos de entrada de nuestro programa.
        f(x) representa la funcion que queremos integrar
        [a,b] es el intervalo donde queremos obtener el valor de la integral
        m son los intervalos que utilizaremos.
"""
x=Symbol('x')
f = 1/(1+x**2)

a=-4
b=4
m=10

"""
    A partir de aqui, comienza el codigo de nuestro programa.
    Primero de todo, calcularemos el valor de h y los nodos que utilizaremos.
"""
n=2*m

h=Decimal((b-a))/Decimal(n)

nodos=[]
for i in range(n+1):
    nodos.append(a+i*h)
    """print(nodos[i])"""

"""
    A continuacion, calcularemos el valor de la integral.
"""
suma1=0
suma2=0
for i in range(1,m+1):
    suma1=suma1+4*f.subs(x,nodos[2*i-1])
    if(i>=2):
        suma2=suma2+2*f.subs(x,nodos[2*i-2])
"""print("\n")
print(f.subs(x,nodos[0]))
print(suma1)
print(suma2)
print(f.subs(x,nodos[n]))

print("\n")
print(h)
print(h/3)
print(f.subs(x,nodos[0])+suma1+suma2+f.subs(x,nodos[n]))"""
integral=(h/3)*(f.subs(x,nodos[0])+suma1+suma2+f.subs(x,nodos[n]))

print(integral)

\end{lstlisting}

\section{Programa 3- Aproximación de Romberg}

\begin{lstlisting}
	#Librerias
import numpy as num
import scipy as sci
from numpy.polynomial import polynomial as pol

def rkj(f,a,b,k,j):
    if(j == 0):
        h = (b-a)/(2**k)
        parcial = 0
        for i in range (2**k - 1):
            parcial = parcial + f(a+i*h)

        res = (h/2)*(f(a) + 2*parcial +f(b))

    else:
        res = rkj(f,a,b,k,j-1) + (1/(4**j-1))*(rkj(f,a,b,k,j-1) - rkj(f,a,b,k-1,j-1))

    return res;



#k = n
def romberg(f,a,b,n):
    aprox = rkj(f,a,b,n,n)
    return aprox



f = lambda x: num.log(x)
a=1
b=2
n=10

res = romberg(f,a,b,n)
print("\n El valor de la aproximacion por el metodo de Romberg es:", res)
\end{lstlisting}


\end{document}